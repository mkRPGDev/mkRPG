\documentclass[a4paper, 11pt]{beamer}

\usepackage{amsfonts, amsmath, amssymb}
\usepackage[utf8]{inputenc}
\usepackage[T1]{fontenc}
\usepackage{minted}
\usepackage{changepage}

\author{Rémi Oudin}
\title{Parsing XML Files}
\date{}
\institute{ENS Cachan}

\usetheme{Hannover}
\usecolortheme{whale}

\setbeamertemplate{navigation symbols}{
    \usebeamerfont{footline}
    \usebeamercolor[fg]{footline}
    \hspace{1em}
    \insertframenumber/\inserttotalframenumber
}

\AtBeginSection[]{
    \begin{frame}
        \frametitle{Contents}
        \tableofcontents[currentsection]
    \end{frame}
}

\begin{document}

\begin{frame}
    \titlepage
\end{frame}

\section*{Introduction}

\begin{frame}
    \frametitle{\secname}
    \begin{itemize}
        \item A game is described by XML files
        \item Need to parse and get the whole data in a quite safe way
        \item Create all instances of a game with  the parsed data.
    \end{itemize}
\end{frame}

\begin{frame}
    \frametitle{\secname~: What is needed.}
    \begin{itemize}
        \item Define a format for the XML files that is compliant with the
            expectations.
        \item Have it general enough for description, and usable in order to
            generate the instances.
        \item Use the good data structures in order to keep parsed data.
        \item Generate instances.
    \end{itemize}
\end{frame}

\section{XML description files}

\begin{frame}
    \frametitle{What needs to be defined}
    \begin{itemize}
        \item Interactions~:reaction to the keys
        \item Actions~: Link between interactions and the world, reaction to
            collision,\dots
        \item Objects~: Entities, objects, maps, cells,\dots
    \end{itemize}
\end{frame}


\subsection{Actions}

\begin{frame}
    \frametitle{\subsecname}
    \begin{figure}
        \begin{adjustwidth}{0.1\textwidth}{}
            \scriptsize{\inputminted{xml}{test_files/actions.xml}}
        \end{adjustwidth}
        \caption{An Action XML tag}
    \end{figure}
    \begin{small}
        \begin{adjustwidth}{-0.1\textwidth}{}
            \begin{description}
                \item[Event] is the interaction that triggers this action.
                \item[Order] is the orders to be executed~:
                    \begin{description}
                        \item[Target] is the target of the action
                        \item[Param] is the parameter to be modified by the
                            action.
                        \item[Value] is the code to be runned. (Yes, by
                            \mintinline{python3}{eval})
                    \end{description}
            \end{description}
        \end{adjustwidth}
    \end{small}
\end{frame}

\subsection{Interactions}

\begin{frame}
    \frametitle{\subsecname}
    \begin{figure}
        \begin{adjustwidth}{0.1\textwidth}{}
            \scriptsize{\inputminted{xml}{test_files/interactions.xml}}
        \end{adjustwidth}
        \caption{An Interaction XML tag}
    \end{figure}
    \begin{small}
        \begin{description}
            \item[Key] is the \emph{keycode} associated to the interaction
            \item[Target] is the target of the interaction~: the player, the
                map,\dots
            \item[Event] is the event to be triggered by the action
        \end{description}
    \end{small}
\end{frame}

\subsection{Entities}

\begin{frame}
    \frametitle{\subsecname}
    \begin{figure}
        \begin{adjustwidth}{0.1\textwidth}{}
            \scriptsize{\inputminted{xml}{test_files/entity.xml}}
        \end{adjustwidth}
        \caption{An Entity XML tag}
    \end{figure}
    \begin{small}
        \begin{adjustwidth}{-0.1\textwidth}{}
            \begin{description}
                \item[Ident] is the numerical identifier of the entity
                \item[Params] describes the parameters of an entity~:
                    \begin{description}
                        \item[Picture] an identifier corresponding to the image
                            representing the character.
                        \item[X, Y] are the starting coordinates
                        \item[Map] is the map on which the entity should be
                            when starting a game.
                    \end{description}
            \end{description}
        \end{adjustwidth}
    \end{small}
\end{frame}

\section{Parsing XML Files}

\subsection{Parsing with Python}

\begin{frame}
    \frametitle{\subsecname}
    \begin{block}{Library}
        Use the python library \mintinline{python3}{xml}, more precisely
        the module \mintinline{python3}{xml.etree.ElementTree}
    \end{block}
    The parser is integrated, most of the work consisted in organizing
    the parsing, and finding the good data structures.
    A big work also was to keep a general enough parsing.
\end{frame}

\subsection{Formatting data}

\begin{frame}[fragile]
    \frametitle{\subsecname}
    \begin{alertblock}{First try}
        Use \mintinline{python3}{dict}. However, dicts iterates in an arbitrary order,
        so the \verb|id|s weren't the same client-side and server-side
    \end{alertblock}

    \pause

    \begin{alertblock}{Second try}
        \mintinline{python3}{dict} and \verb|id|s in the xml. It still fails
        since some data was created at runtime with different \verb|id|s.
    \end{alertblock}

    \pause

    \begin{block}{Third try}
        Use \mintinline{python3}{collection.OrderedDict}s in order to solve
        all the issues.
    \end{block}
\end{frame}

\subsection{Keep a general parsing}

\begin{frame}
\end{frame}
\end{document}
